\documentclass{article}
\begin{document}
\setlength{\parindent}{0pt}

There exist two PDFs:

\[ h(Z,A) \]

and 

\[ \ell(Z,A) \]

The bidirectional translation between $h(Z,A)$ and $\ell(Z,A)$ is given by:

\[ P(A \rightarrow A') \]

such that:

\begin{equation}
\ell(Z',A') = \int P(A \rightarrow A')\, h(Z,A)\,\, dA
\end{equation}

and similarly:

\begin{equation}
h(Z',A') = \int P(A \rightarrow A')\, \ell(Z,A)\,\, dA
\end{equation}

For each PDF there the expectation value in $A$ is: 

\begin{equation}
\bar{A}_h = \int A\, h(Z,A)\, dA
\end{equation}

and:

\begin{equation}
\bar{A}_\ell = \int A\, \ell(Z,A)\, dA
\end{equation}

And for these particular PDFs it is known empirically that the sum of these expectation values equals a known constant, $A_{tot}$:

\begin{equation}
\bar{A}_{h} + \bar{A}_{\ell} = A_{tot}
\end{equation}

Now say that one PDF is modified:

\begin{equation}
h'(Z,A) = h(Z,A) + \sigma(Z,A)
\end{equation}

where $\sigma(Z,A)$ is a modifier that is selected such that the normalization of $h'(Z,A)$ is maintained. This modifier might for example be some statistical resampling process that increases values in $h(Z,A)$ in some regions and decreases it in some regions. This additive modifier is similar to that of the quantile function of a Gaussian distribution. \\

And then the other PDF is regenerated from the modified PDF: 

\begin{equation}
\ell'(Z,A) =  \int P(A' \rightarrow A)\, h'(Z',A')\,\, dA'
\end{equation}

Now show that the sum of the expectation values of these modified PDFs is equal to that of the original PDFs: 

\begin{equation}
\bar{A}_{h}' + \bar{A}_{\ell}' = A_{tot}
\end{equation}

Calculate $\bar{A}_{h}'$:

\[ \bar{A}_h' = \int A\, h'(Z,A)\, dA \] 

\[ \bar{A}_h' = \int A\, h(Z,A) + \sigma(Z,A) \, dA \] 

\[ \bar{A}_h' = \int A\, h(Z,A)\, dA + \int A\, \sigma(Z,A)\, dA \] 

\[ \bar{A}_h' = \bar{A}_{h} + \int A\, \sigma(Z,A)\, dA \] 

\begin{equation}
\bar{A}_h' = \bar{A}_{h} + \Delta_h
\end{equation}

where:

\begin{equation}
\Delta_h = \int A\, \sigma(Z,A)\, dA
\end{equation}

Calculate $\bar{A}_{\ell}'$:

\[ \bar{A}_{\ell}' = \int A\, \ell'(Z,A)\, dA \]

\[ \bar{A}_{\ell}' = \int A\, \int P(A' \rightarrow A)\, h'(Z',A')\,\, dA'\, dA \]

\[ \bar{A}_{\ell}' = \int A\, \int P(A' \rightarrow A)\, \Big[h(Z',A') + \sigma(Z',A')\Big]\,\, dA'\, dA \]

\[ \bar{A}_{\ell}' = \int A\, \Bigg[ \int P(A' \rightarrow A)\, h(Z',A')\,dA' + \int P(A' \rightarrow A)\, \sigma(Z',A') )\,\, dA'  \Bigg]\, dA \]

\[ \bar{A}_{\ell}' = \int A\, \Bigg[ \ell(Z,A) + \int P(A' \rightarrow A)\, \sigma(Z',A') )\,\, dA'  \Bigg]\, dA \]

\[ \bar{A}_{\ell}' = \Bigg[ \int A\, \ell(Z,A)\,dA + \int A\, \int P(A' \rightarrow A)\, \sigma(Z',A') )\,\, dA'\, dA \Bigg] \]

\[ \bar{A}_{\ell}' = \Bigg[ \int A\, \ell(Z,A)\,dA + \int A\, \int P(A' \rightarrow A)\, \sigma(Z',A') )\,\, dA'\, dA \Bigg] \]

\[ \bar{A}_{\ell}' = \Bigg[ \bar{A}_{\ell} + \int A\, \int P(A' \rightarrow A)\, \sigma(Z',A')\,\, dA'\, dA \Bigg] \]

\begin{equation}
\bar{A}_{\ell}' = \bar{A}_{\ell} + \Delta_\ell
\end{equation}

where:

\begin{equation}
\Delta_\ell = \int A\, \int P(A' \rightarrow A)\, \sigma(Z',A')\,\, dA'\, dA
\end{equation}

So then: 

\begin{equation}
\bar{A}_{h}' + \bar{A}_{\ell}' = \bar{A}_{h} + \Delta_h + \bar{A}_{\ell} + \Delta_\ell
\end{equation}

In order to have a non-trivial answer, it must be shown that $\Delta_h$ and $\Delta_\ell$ are of equal and opposite magnitude:

\begin{equation}
\Delta_h - \Delta_\ell = 0
\end{equation}

Based on the definitions of $\Delta_h$ and $\Delta_\ell$, for to be true then it must also be true that: 

\begin{equation}
\int P(A' \rightarrow A)\, \sigma(Z',A')\,\, dA' = -\sigma(Z,A)
\end{equation}

The effect of acting $P(A' \rightarrow A)$ on $\sigma(Z',A')$ must be derived: 

\[ h(Z',A') = h'(Z',A') - \sigma(Z',A') \]

\[ \ell(Z,A) = \int P(A' \rightarrow A)\, h(Z',A')\,\, dA' \]

\[ \ell(Z,A) = \int P(A' \rightarrow A)\, [h'(Z',A') - \sigma(Z',A')]\,\, dA' \]

\[ \ell(Z,A) = \int P(A' \rightarrow A)\, h'(Z',A')\, dA' - \int P(A' \rightarrow A)\, \sigma(Z',A')\, dA' \]

\[ \int P(A' \rightarrow A)\, \sigma(Z',A')\, dA' = \int P(A' \rightarrow A)\, h'(Z',A')\, dA'\,\, - \ell(Z,A) \]

And so the value of $\Delta_\ell$ is:

\[ \Delta_\ell = \int A\, \Bigg[ \int P(A' \rightarrow A)\, h'(Z',A')\, dA'\,\, - \ell(Z,A) \Bigg]  \, dA \]

\[ \Delta_\ell = \int A\, \int P(A' \rightarrow A)\, h'(Z',A')\, dA'\,dA \,\, - \int A\, \ell(Z,A) \, dA \]

\[ \Delta_\ell = \int A\, \ell(Z,A)\,dA \,\, - \bar{A}_\ell \]

\begin{equation}
\Delta_\ell =\bar{A}_\ell  - \bar{A}_\ell = 0
\end{equation}


Now the value of $\Delta_h$ must be derived:

\[ \sigma(Z,A) = h'(Z,A) - h(Z,A) \]

\[ \Delta_h = \int A\, \sigma(Z,A)\, dA \]

\[ \Delta_h = \int A\, [h'(Z,A) - h(Z,A)]\, dA \]

\[ \Delta_h = \int A\, h'(Z,A)\, dA - \int A\, h(Z,A)\, dA \]

\begin{equation}
\Delta_h = \bar{A}_h' - \bar{A}_h
\end{equation}

So then: 

\[ \bar{A}_{h}' + \bar{A}_{\ell}' = \bar{A}_{h} + \Delta_h + \bar{A}_{\ell} + \Delta_\ell \]

\[ \bar{A}_{h}' + \bar{A}_{\ell}' = \bar{A}_{h} + \bar{A}_h' - \bar{A}_h + \bar{A}_{\ell} + 0 \]

\[ \bar{A}_{h}' + \bar{A}_{\ell}' = \bar{A}_h' + \bar{A}_{\ell} \]

\[ \bar{A}_{h}' + \bar{A}_{\ell}' = \bar{A}_h' + \bar{A}_{\ell} \]















\end{document}